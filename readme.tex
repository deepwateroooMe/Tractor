% Created 2023-05-10 Wed 16:42
\documentclass[9pt, b5paper]{article}
\usepackage{xeCJK}
\usepackage[T1]{fontenc}
\usepackage{bera}
\usepackage[scaled]{beraserif}
\usepackage[scaled]{berasans}
\usepackage[scaled]{beramono}
\usepackage[cache=false]{minted}
\usepackage{xltxtra}
\usepackage{graphicx}
\usepackage{xcolor}
\usepackage{multirow}
\usepackage{multicol}
\usepackage{float}
\usepackage{textcomp}
\usepackage{algorithm}
\usepackage{algorithmic}
\usepackage{latexsym}
\usepackage{natbib}
\usepackage{geometry}
\geometry{left=1.2cm,right=1.2cm,top=1.5cm,bottom=1.2cm}
\usepackage[xetex,colorlinks=true,CJKbookmarks=true,linkcolor=blue,urlcolor=blue,menucolor=blue]{hyperref}
\newminted{common-lisp}{fontsize=\footnotesize} 
\author{deepwaterooo}
\date{\today}
\title{ET 框架拖拉机项目试改装}
\hypersetup{
  pdfkeywords={},
  pdfsubject={},
  pdfcreator={Emacs 28.2 (Org mode 8.2.7c)}}
\begin{document}

\maketitle
\tableofcontents


\section{双副牌双升108 张卡牌游戏}
\label{sec-1}
\begin{itemize}
\item 【游戏可试玩程序】:放在Release/Tractor.exe. Windows 用户大家可以下载试玩儿体验一下。
\item 昨天晚上找见了别人几年前就开发出来的卡五星麻将,所以写麻将游戏的想法就被恶杀在摇篮中。现在再写什么好呢?就只能写【双升拖拉机】了,就是两副牌108 张来打的拖拉机。现已经 ios iPhone 上有的双升游戏,可能搜索一下设计,写安卓版的双升了,看下能否套用ET 框架,写成四人网络【客户端与服务器双热更新的】网络游戏
\item 现在先搜索必要的框架设计,出版规则比大小算法之类的。
\item 【服务器与客户端的同步】:尤其是在分四人牌后,亮主拖底的时候,谁先亮,亮什么主,顺序重要,结果重要。【 ET 框架有专用的游戏服,由游戏服来状态同步】在本程序中,采用的是服务器保存所有的状态,处理所有的逻辑。比如,客户端在点击亮主后,做的事情就是发一个消息给服务器,不做任何显示操作,等待服务器传来亮主的消息后再显示
\begin{itemize}
\item 【发牌,公正性】:随机分牌。第一步就是要发牌。需要做到一个完全随机的发牌,就要保证每张牌发到每个玩家手里的概率都是一样的,而且牌的顺序是等概率随机打乱的。程序中采用的是如下的发牌算法(感谢Dr.Light提供):假如有两幅牌,编号从1到108,首先随机选出一个,并且将牌发给玩家,然后将这个编号的牌与108号牌交换编号,那么剩下的牌就是从1到107号。于是再从中选出一个,重复以上的过程,这样一来,算法的复杂度就是O(n)。
\end{itemize}
\end{itemize}
\section{主要想要改进的地方:前辈十年前开发出的游戏,游戏整体【差强人意】}
\label{sec-2}
\begin{itemize}
\item \textbf{【界面设计:】} 八十年前没有ET 框架,不知道原作者是如何设计这个游戏的。感觉游戏的整体走桌面游戏风,要把菜单设计等改成【手游风格】。
\item \textbf{【逻辑设计,用户意愿:】} 不【尊重用户游戏配置选择】的游戏,永远是固执不受欢迎的。【游戏逻辑、玩法】需要能够给用户留足选择配置空间,如下:
\begin{itemize}
\item 现有的游戏规则:再找了理解一下:现找到原项目中提到这些。但作为程序员,不把它所有相关逻辑读懂,都感觉不明白它说的狠多是什么意思
\end{itemize}
\end{itemize}

\includegraphics[width=.9\linewidth]{./pic/readme_20230510_160604.png}

\begin{center}
\begin{tabular}{ll}
\textbf{2 为常主} & 是常主,就常主会比较多,否则常主少,尤其打王的时候\\
\textbf{5 10 K 必打} & 因为比较难打\\
\textbf{单 J 勾到6, 双 JJ 勾到 2} & 开历史倒车,增加游戏的无穷乐趣:惊险刺激:逢对家对J, 谁不想把对方废到6 或是2?\\
 & 逢自家打到J, 会想要被对方废到6 或是2?\\
\textbf{逢 J 必打} & 因为上面不能言说的【惊险刺激】,不给任何双方逃跑的好玩机会\\
\textbf{小光升1 级,大光连升3 级} & 不满 40 算小光,0 分为大光头。。。连升三级\\
\textbf{捡分方扣底,底分翻倍} & 单扣乘2, 又扣乘 4, 拖拉机扣牌数乘 2.\\
 & 如此,才能让捡分方快速超越,以风马牛不能及之势火箭升级。。。\\
\end{tabular}
\end{center}
\begin{itemize}
\item \textbf{【关于J Q】}, 游戏设置: 
\begin{itemize}
\item 增强的规则:在一些地方流行一J到底、Q到半的玩法。
\item 庄家在打J时,如果下台,并且最后一把被J抠底,那么此庄家再上台时将从2开始打。
\item 庄家在打Q时,如果下台,并且最后一把被Q抠底,那么此庄家再上台时将从6开始打。
\end{itemize}
\item \textbf{【流局规则】} :如果摸牌时无人亮主,那么有两种选择,一是【流局】,重新发牌,二是【揭底】,将八张底全部揭开,选第三张牌的花色作为主花色。感觉流局设置不太好玩,浪费时间。
\item \textbf{【扣底规则】} :计算机在扣牌时,有三种扣牌算法可供选择:这里说的应该是,机器人应对其它三个玩家时的扣底规则,放8 张底牌,如何放原则。
\begin{itemize}
\item 1.激进算法,以扣绝一门为主要目标
\item 2.中庸算法,以不扣分(5分除外)为主要目标
\item 3.保守算法,以不扣分不扣对为主要目标
\end{itemize}
\item \textbf{【亮主规则】} :可以选择是否允许自反,加固和亮无主
\item \textbf{【亮牌规则】} :(注:以打8为例) \textbf{【这里有狠大的游戏逻辑改进,和游戏用户体验提升空间】}
\begin{itemize}
\item 在发牌过程中,第一次亮出的8的花色作为主牌花色。
\item 有以下几种情况可改变或加强主牌花色:
\begin{itemize}
\item 自保
\item 反主
\end{itemize}
\item 以上后三条以先出现者为准。
\item 若发牌结束仍无人亮牌,则以底牌第一张的花色作为主牌花色。
\end{itemize}
\item \textbf{【升级规则】}
\begin{itemize}
\item 闲家得0分为大光,庄家升三级。
\item 闲家得分小于40分为小光,庄家升二级。
\item 闲家得分大于等于40分且小于80分时,庄家升一级。
\item 闲家得分大于等于80分且小于120分时,闲家上台。
\item 闲家得分大于等于120分且小于160分时,闲家上台且升一级。
\item 闲家得分大于等于160分且小于200分时,闲家上台且升二级。
\item 闲家得分大于等于200分时,闲家上台且升三级。
\end{itemize}
\item \textbf{【打牌规则】} :(注:以打10为例) \textbf{出牌时同等大小的牌以先出者为大。}
\begin{itemize}
\item \textbf{同门花色的大牌可以联出,称作“甩牌”} 如:
\item 副牌中:AAK,AKK,AQQJJ,
\item 98844(若其他家中无人有能大过一张9,和一对8,和一对4的牌)。
\item \textbf{若首家试图联出的牌并非都是大牌时,则其必须出欲联出的牌中的最小牌。} 如:
\begin{itemize}
\item 首家试图联出98844时,若其余某家有此花色的J,则首家必须出9,若其余某家有此花色的QQ或55,则首家必须出44。
\item 首家出对牌时,其余家有对牌必须出对牌(包括拖拉机中的对牌)
\item 首家出拖拉机时,其余家有拖拉机必须出拖拉机,若无拖拉机,则必须出对牌,无对牌时才能出其它牌。
\end{itemize}
\item \textbf{首家出某花色副牌时,其余家无此门花色时,可出主牌,称为“毙”。} 若首家出的牌中有拖拉机或对牌,毙牌时所出的牌必须是主牌,且其拖拉机的数目不得少于首家出的牌中的拖拉机的数目,对牌的数目也不得少于首家出的牌中的对牌的数目,否则被视为垫牌。
\item *出现多家毙牌时,毙牌的大小以毙牌中的拖拉机和对牌大小为准,大的称为“盖毙”。*如:
\begin{itemize}
\item 主牌998872可毙副牌AK5544,但不能毙副牌AA5544
\item 主牌977可毙副牌544,主牌884可盖毙
\item 主牌977可毙副牌567,主牌884不能盖毙
\end{itemize}
\end{itemize}
\item \textbf{【抠底规则】} :
\begin{itemize}
\item 以单张牌抠底时底牌分数乘二。
\item 以对牌牌抠底时底牌分数乘四。
\item 以拖拉机抠底时底牌分数乘八 \textbf{【应该是拖拉机张数乘以2】} 。因为大拖拉机可以三对四对。。。或留底甩牌,只要能大。。。
\end{itemize}
\item \textbf{【拖拉机的构成】} :(注:以打10为例)
\begin{itemize}
\item \textbf{凡大小顺序相邻且花色相同的联对均构成拖拉机} ,如:
\begin{itemize}
\item KKQQ,JJ99,554433;
\end{itemize}
\item \textbf{主牌中凡大小顺序相邻联对均构成拖拉机} ,如:
\begin{itemize}
\item 一对小王带一对主10,一对主10带一对副10
\item 一对副10带一对主牌A,一对主10带一对副10及一对主牌A
\end{itemize}
\item 以下各例均不是拖拉机:
\begin{itemize}
\item 554,544,5533,JJQQ,两对副10,JJ1010,AA22
\end{itemize}
\end{itemize}
\item \textbf{【牌的大小顺序】} :现在游戏框架设计,束缚了用户的【2 为常主】的配置选择,算法,数据结构等,需要重构
\begin{itemize}
\item 以打10为例
\item 主牌从大至小依次为:
\begin{itemize}
\item 大王,小王,主10,副10,A,K,Q,J,9,8,7,6,5,4,3,2
\end{itemize}
\item 副牌从大至小依次为:
\begin{itemize}
\item A,K,Q,J,9,8,7,6,5,4,3,2
\end{itemize}
\end{itemize}
\item \textbf{【轮庄规则】} :为创造出好玩儿的玩法,这里是可以优化改进的。对家的本意是,两人合作,快速升级,所以需要两者配合。不需要,或可以配置不规定严格的顺序,给予他们无数无限合作可能,给予对方继续反副反主的机会,增加游戏趣味。
\begin{itemize}
\item 开局中,双方争庄,先亮者为庄家。
\item 庄家升级时,下一副牌由其对家当庄家。
\item 闲家上台时,下一副牌由此副牌的庄家的下家当庄家。
\end{itemize}
\item 其它这里没有列出来的,主要是我现在还不曾了解那些是在说什么,比如下面网络上提到过的:提供六种配置选项: \textbf{【允许自反】,允许对家保,允许反无将,A 必打} (是为什么呢,K 易跑光,不好捡分?)等
\item \textbf{【点击触屏、用户交互的性能优化】} :需要优化。玩家就算玩得不久,一直点鼠标,也是痛苦的事。需要AI 辅助,智能帮助用户出牌,让鼠标点击、选牌聪敏、反应快。
\begin{itemize}
\item 原游戏应该是桌面游戏,所以会有快捷键设置。但手游,就需要自己将触屏设置优化出来
\end{itemize}
\item \textbf{【逻辑设计,用户意愿:】}: 逻辑上,为能实现以上种种好玩玩法,游戏逻辑需要 \textbf{规定,约束严格的反牌规则:从高到低为【王黑红梅方】} ,就是别人叫方块的主,其它都可以反,但若是已经反到黑桃,接下来就只能反王或说是常主。允许捡分方按照以上规则反牌,这样才给给予捡分方底牌放 80 分,拖拉机扣底,火箭升级的机会。规则明确,公正。现游戏中一个【“流局”】界面,抹杀了这一切好玩儿的过程与结果,太不好玩了。。。游戏界面,也需要必要的文字提示等,帮助玩家理解游戏中的这些好玩儿规则,让玩家上瘾。。。
\end{itemize}
\section{游戏整体【差强人意】现游戏试玩中抓到的【BUG:】如下}
\label{sec-3}
\begin{itemize}
\item 不考虑现代大型网络游戏的双端热更新机制。现在游戏的热更新实在是必备。游戏整体,逻辑相对完整,提供了完整的AI 辅助,主要只是提供了 \textbf{牌面的背景图、游戏桌面背景图、背景音效等配置} 。但 \textbf{游戏逻辑单一固定,不好玩。}
\begin{itemize}
\item 现有的游戏中已经配置如下:只有算法,以及游戏性能需要优化
\end{itemize}
\end{itemize}
\subsection{游戏}
\label{sec-3-1}
\begin{itemize}
\item \textbf{【开始新游戏】} :开始新的游戏,从2打起
\item \textbf{【暂停游戏】} :可以暂停游戏,再点击此菜单将继续游戏
\item \textbf{【保存牌局】} :将游戏的状态保存起来,包括各家在打几,庄家是谁,目前打几
\item \textbf{【读取牌局】} :读取保存的牌局,重新发牌
\end{itemize}
\subsection{设置}
\label{sec-3-2}
\begin{itemize}
\item \textbf{【游戏速度】} :可以设置游戏的每个步骤的速度,左边为快速,右边为慢速
\item \textbf{【牌面图案】} :有三种图案可供选择,你也可以选择自己制作的牌面
\item \textbf{【牌背图案】} :有三种背面图可供选择
\item \textbf{【牌桌图案】} :可以选择背景图案,图片大小为固定大小,如果不是这个尺寸,图片将进行缩放
\item \textbf{【背景音乐】} :可以设置打牌时的背景音乐,支持wav、mp3、midi三种音乐格式,可随机、循环播放
\item \textbf{【游戏规则】} :可以设置必打、增强(一J到底、一Q到半)、揭底、扣底、亮主等规则。 \textbf{【缺点:】} 对新玩家来说,这些概念不明确,需要游戏界面提醒
\item \textbf{【机器人罗伯特】} :这个机器人可以代替您打牌。 \textbf{【想把这个更多的用在,手游辅助触屏点击时】}
\end{itemize}
\subsection{工具}
\label{sec-3-3}
\begin{itemize}
\item \textbf{【拖拉机伴侣】} :使用这个工具可以制作您自己的牌面,将您的数码照片嵌入到游戏中【这个可能有点儿多余】。但仍可以手游上试执行。
\end{itemize}
\section{主要【BUG:】}
\label{sec-4}
\begin{itemize}
\item 对游戏整体的玩家用户体验如此,但并不是说我就真的狠懂这个游戏项目。实际上,我还没能真正学习这个项目,甚至它底层的算法动态库的连接等,都是我需要从这个十年前的项目中学习的地方。借他山之石,为自己的游戏所用。
\item 现在抓到的主要 bug 如下截图:
\end{itemize}

\includegraphics[width=.9\linewidth]{./pic/readme_20230509_230111.png}

\includegraphics[width=.9\linewidth]{./pic/readme_20230509_232252.png}

\includegraphics[width=.9\linewidth]{./pic/readme_20230510_014418.png}

\includegraphics[width=.9\linewidth]{./pic/readme_20230510_015324.png}

\includegraphics[width=.9\linewidth]{./pic/readme_20230510_033444.png}

\includegraphics[width=.9\linewidth]{./pic/readme_20230510_042818.png}

\includegraphics[width=.9\linewidth]{./pic/readme_20230510_043722.png}

\begin{itemize}
\item 【牌的逻辑OOD/OOP】设计:三个类,对应单张,拖拉机(对子是长度为1 的拖拉机),和混合单张与拖拉机
\item 简易版设计原理:模拟拖拉机(升级)玩法;
\begin{itemize}
\item 1.创建两副牌的集合:HashMap
\item 2.创建纸牌:四个花色共108张♦ ♣ ♥ ♠
\item 3.创建poker的ArrayList操作集合
\item 4.创建亮主牌的操作
\item 5.将所有牌放入牌盒中
\item 6.创建四个玩家与底牌的集合:HashSet wj1,wj2,wj3,wj4,dipai
\item 7.洗牌
\item 8.发牌操作
\item 9.创建看牌方法
\item 10.调用方法看牌
\end{itemize}
\item 安桌上的游戏现在是这样的:还要再写一个吗?【活宝妹就是一定要嫁给亲爱的表哥!!!】还是说更为完善或是好玩儿的游戏逻辑?或是UI 视图画面,或是性能表现?反正一定是套用ET 框架写得最容易快速方便。【感觉现在这个截图的UI 长得有点儿丑怪。。】不好看不经典,看了就不想玩儿了。。
\item 因为各处的游戏规则不一样,所以给玩家多点儿自由,自己选择玩法。提供六种配置选项:【允许自反】,允许对家保,2 为常主,允许反无将,五十K 必打,JA 必打等
\end{itemize}
% Emacs 28.2 (Org mode 8.2.7c)
\end{document}