% Created 2023-12-20 Wed 13:37
\documentclass[9pt, b5paper]{article}
\usepackage{xeCJK}
\usepackage[T1]{fontenc}
\usepackage{bera}
\usepackage[scaled]{beraserif}
\usepackage[scaled]{berasans}
\usepackage[scaled]{beramono}
\usepackage[cache=false]{minted}
\usepackage{xltxtra}
\usepackage{graphicx}
\usepackage{xcolor}
\usepackage{multirow}
\usepackage{multicol}
\usepackage{float}
\usepackage{textcomp}
\usepackage{algorithm}
\usepackage{algorithmic}
\usepackage{latexsym}
\usepackage{natbib}
\usepackage{geometry}
\geometry{left=1.2cm,right=1.2cm,top=1.5cm,bottom=1.2cm}
\usepackage[xetex,colorlinks=true,CJKbookmarks=true,linkcolor=blue,urlcolor=blue,menucolor=blue]{hyperref}
\newminted{common-lisp}{fontsize=\footnotesize} 
\author{deepwaterooo}
\date{\today}
\title{ET 框架拖拉机项目源码与设计重构}
\hypersetup{
  pdfkeywords={},
  pdfsubject={},
  pdfcreator={Emacs 29.1 (Org mode 8.2.7c)}}
\begin{document}

\maketitle
\tableofcontents


\section{DefinedConstant | CardCommands | CurrentState}
\label{sec-1}
\begin{minted}[fontsize=\scriptsize,linenos=false]{csharp}
// 程序常量
class DefinedConstant {
    // 时间常量
    internal const int FINISHEDONCEPAUSETIME = 1500; // 每圈暂停时间
    internal const int NORANKPAUSETIME = 5000; // 流局时间
    internal const int GET8CARDSTIME = 1000; // 摸8张底牌的时间
    internal const int SORTCARDSTIME = 1000; // 我的牌排序时间
    internal const int FINISHEDTHISTIME = 2500; // 每局暂停时间
    internal const int TIMERDIDA = 100; // 系统滴答
}
// 命令状态,指示下一步动作
enum CardCommands {
    ReadyCards, // 发牌命令
    DrawCenter8Cards, // 画8张底牌的命令
    WaitingForSending8Cards, // 等待扣底的命令
    DrawMySortedCards,// 排序我的牌的命令
    Pause,// 通用暂停命令
    WaitingShowPass, // 显示流局的命令
    WaitingShowBottom, // 翻底牌的命令
    WaitingForSend, // 等待出牌
    WaitingForMySending, // 等待我出牌的命令
    DrawOnceFinished,// 出完一圈后的命令
    DrawOnceRank,// 出完一局后的命令
    Undefined // 未定义的命令
}
// 保存当前游戏状态的对象
[Serializable]
struct CurrentState {
    internal int OurCurrentRank; // 自己当前的牌局
    internal int OurTotalRound; // 总轮数
    internal int OpposedCurrentRank; // 对方的牌局
    internal int OpposedTotalRound; // 总轮数

    // 当前的【庄家】:未定0, 自己1、对家2、西3、东4
    internal int Master;

    // 当前的【花色】:未定0、红桃1、黑桃2、方块3、梅花4、无主5
    internal int Suit;

    internal CardCommands CurrentCardCommands; // 当前命令
    internal CurrentState(int ourCurrentRank, int opposedCurrentRank, int suit, int master,int ourTotalRound,int opposedTotalRound, CardCommands currentCardCommands) { // tv ..
    }
}
\end{minted}
\section{CurrentPoker : Diamonds | 梅花 | 红桃 | 黑桃}
\label{sec-2}
\begin{itemize}
\item \textbf{【非OOD/OOP 设计的局限】} :当亲爱的表哥的活宝妹想要扩展,用户玩家自己配置游戏规则【2 为常主,与否】时,这个非OOD/OOP 的游戏实现,就把它写死了,得重构。
\item 这个东西重复四遍,什么意思嘛。就是因为这么初始化,没法支持用户配置【2 为常主】
\item 设置为一个广谱的 OneSuit 之类的类,可以实例成四种类型,并且根据2 是否为常主来设置大小
\begin{minted}[fontsize=\scriptsize,linenos=false]{csharp}
#region 方块
// 方块(2,3,4,5,6,7,8,9,10,J,Q,K,A)
        private int[] diamonds = { 0, 0, 0, 0, 0, 0, 0, 0, 0, 0, 0, 0, 0 };
        internal int[] Diamonds {
            get { return diamonds; }
            set { diamonds = value; }
        }
// 不带主的方块
        private int[] diamondsNoRank = { 0, 0, 0, 0, 0, 0, 0, 0, 0, 0, 0, 0, 0 };
        internal int[] DiamondsNoRank {
            get { return diamondsNoRank; }
            set { diamondsNoRank = value; }
        }
// 方块Rank数
        internal int DiamondsRankTotal = 0;
// 方块非Rank数
        internal int DiamondsNoRankTotal = 0;
// 排序的牌型
        internal int[] SortCards = new int[56];
#endregion // 方块
\end{minted}
\item 游戏文件夹 mixed 里埋了狠多 .dll 动态库。这要打算要怎么处理呢?
\item \textbf{【有偏洗牌、算法】} : \textbf{【亲爱的表哥的活宝妹,任何时候,亲爱的表哥的活宝妹就是一定要、一定会嫁给活宝妹的亲爱的表哥!!!爱表哥,爱生活!!!】}
\item 随机洗牌使得“牌洗得太均匀,不够吸引玩家,也不利于厂家让玩家买豆(或者金币)的目标”的不足。本节以斗地主和升级(或者拖拉机)为例,呈现如何有偏洗牌,例如在斗地主游戏中希望能够”人为“控制炸弹出现比例。由于每个游戏的有偏洗牌论述及代码过长,为此对分两小节以分别斗地主和升级的有偏洗牌过程。
\item 上面,这个思路是好的。不知道到时能否实现。就是,亲爱的表哥的活宝妹,故意、人为制造有偏洗牌,使得游戏更好玩儿!
\item Q1: 有偏洗牌的通用框架(或解决思路)是什么?
\begin{itemize}
\item (1) 生成所需的扑克数。
\item (2) 设计抽样规则,抽样生成有偏的牌,然后从牌堆扣除有偏的牌(如升级中的炸弹)。
\item (3) 将有偏向的牌以等概率发给各玩家
\item (4) 对于剩余的牌进行随机排列
\item (5) 将随机排列后的牌发给各玩家,补足各玩家需要的牌数(如斗地主中各玩家需17张)
\end{itemize}
\item 这个算法的网页列一下: \url{https://zhuanlan.zhihu.com/p/363599902}
\item \textbf{【斗地主游戏有偏洗牌】} 的框架可以被用来解决 \textbf{【拖拉机有偏洗牌】} ,其中代码(或逻辑)需要调整之处在于:
\begin{itemize}
\item (1) 拖拉机游戏有偏牌的产生规则。
\item (2)拖拉机游戏的对子统计,其中对子是指两张具有相同颜色和点数的牌对。
\item (3)拖拉机游戏中牌的排序。
\end{itemize}
\item 其中问题(2)-(3)属于牌的统计与显示,问题(1)才是核心。如何接下来聚焦如何解决问题(1)。
\begin{itemize}
\item 参考来自于:\url{https://zhuanlan.zhihu.com/p/363677920} 可是是可恶的 python 编程。。。
\end{itemize}
\item 网络上有个某个主程它总结的扑克牌游戏相关,但是 \textbf{它应该也是网络洗脑来着,写得、总结得极为前端 html 化,所以感觉难度相应地降低了狠多。但是对比如 10 款、20 款扑克牌游戏基本模块的拆分、与总结、归纳、概括算是到位的;但经典精华的地方,总结里全部跳过了;亲爱的表哥的活宝妹,应该借助这个思路、与他人的总结来想,在手游端【安卓,苹果】,亲爱的表哥的活宝妹可以设计与实现哪些、哪类?能否如本文的 html 小前端主程总结过的,找出,亲爱的表哥的活宝妹自己,可以开发的潜能与方向?} \url{https://zhuanlan.zhihu.com/p/173703104}
\item \textbf{【亲爱的表哥的活宝妹,任何时候,亲爱的表哥的活宝妹就是一定要、一定会嫁给活宝妹的亲爱的表哥!!!爱表哥,爱生活!!!】}
\end{itemize}
\section{ET7 框架拖拉机游戏设计,源码分析与重构, 或是【参考项目斗地主里的设计】}
\label{sec-3}
\subsection{【GamerComponent】玩家组件管理类}
\label{sec-3-1}
\begin{itemize}
\item 管理所有一个房间的玩家:是对一个房间里四个玩家的(及其在房间里的坐位位置)管理(分东南西北)。可以添加移除玩家。今天晚上来弄这一块儿吧。
\item 组件:是提供给房间用,用来管理游戏中每个房间里的最多三个当前玩家
\end{itemize}
\begin{minted}[fontsize=\scriptsize,linenos=false]{csharp}
public class GamerComponent : Entity, IAwake { // 它也有【生成系】
    private readonly Dictionary<long, int> seats = new Dictionary<long, int>();
    private readonly Gamer[] gamers = new Gamer[4]; 
    public Gamer LocalGamer { get; set; } // 提供给房间组件用的:就是当前玩家。。。
    // 添加玩家
    public void Add(Gamer gamer, int seatIndex) {
        gamers[seatIndex] = gamer;
        seats[gamer.UserID] = seatIndex;
    }
    // 获取玩家
    public Gamer Get(long id) {
        int seatIndex = GetGamerSeat(id);
        if (seatIndex >= 0) 
            return gamers[seatIndex];
        return null;
    }
    // 获取所有玩家
    public Gamer[] GetAll() {
        return gamers;
    }
    // 获取玩家座位索引
    public int GetGamerSeat(long id) {
        int seatIndex;
        if (seats.TryGetValue(id, out seatIndex)) 
            return seatIndex;
        return -1;
    }
    // 移除玩家并返回
    public Gamer Remove(long id) {
        int seatIndex = GetGamerSeat(id);
        if (seatIndex >= 0) {
            Gamer gamer = gamers[seatIndex];
            gamers[seatIndex] = null;
            seats.Remove(id);
            return gamer;
        }
        return null;
    }
    public override void Dispose() {
        if (this.IsDisposed) 
            return;
        base.Dispose();
        this.LocalGamer = null;
        this.seats.Clear();
        for (int i = 0; i < this.gamers.Length; i++) 
            if (gamers[i] != null) {
                gamers[i].Dispose();
                gamers[i] = null;
            }
    }
}
\end{minted}
\subsection{Gamer | GamerAwakeSystem}
\label{sec-3-2}
\begin{minted}[fontsize=\scriptsize,linenos=false]{csharp}
[ObjectSystem]
public class GamerAwakeSystem : AwakeSystem<Gamer,long> {
    protected override void Awake(Gamer self, long id) {
        self.Awake(id);
    }
}
// 房间玩家对象
public sealed class Gamer : Entity, IAwake<long> {
    // 用户ID(唯一)
    public long UserID { get; private set; }
    // 玩家GateActorID
    public long PlayerID { get; set; }
    // 玩家所在房间ID
    public long RoomID { get; set; }
    // 是否准备
    public bool IsReady { get; set; }
    // 是否离线
    public bool isOffline { get; set; }

    public void Awake(long id) {
        this.UserID = id;
    }
    public override void Dispose() {
        if (this.IsDisposed) return;
        base.Dispose();
        this.UserID = 0;
        this.PlayerID = 0;
        this.RoomID = 0;
        this.IsReady = false;
        this.isOffline = false;
    }
}
\end{minted}
\subsection{Card}
\label{sec-3-3}
\begin{minted}[fontsize=\scriptsize,linenos=false]{csharp}
public partial class Card : IEquatable<Card> {    // 牌类
    public bool Equals(Card other) { // 数字与花型 
        return this.CardWeight == other.CardWeight && this.CardSuits == other.CardSuits;
    }
    public string GetName() { // 获取卡牌名
        return this.CardSuits == Suits.None ? this.CardWeight.ToString() : $"{this.CardSuits.ToString()}{this.CardWeight.ToString()}";
    }
}
\end{minted}
\subsection{}
\label{sec-3-4}
\subsection{TractorRoomComponent: 主要是里面嵌套一个 TractorInteractionComponent 组件}
\label{sec-3-5}
\begin{minted}[fontsize=\scriptsize,linenos=false]{csharp}
// [ObjectSystem] // AwakeSystem : AwakeSystem<TractorRoomComponent> {
public class TractorRoomComponent : Entity, IAwake {
    private TractorInteractionComponent interaction;
    private Text multiples;
    public readonly GameObject[] GamersPanel = new GameObject[4];
    public bool Matching { get; set; }
    public TractorInteractionComponent Interaction { // 去找:组件里套组件,要如何事件机制触发生成?
        get {
            if (interaction == null) {
                UI uiRoom = this.GetParent<UI>();
                UI uiInteraction = TractorInteractionFactory.Create(UIType.TractorInteraction, uiRoom);
                interaction = uiInteraction.GetComponent<TractorInteractionComponent>();
            }
            return interaction;
        }
    }
    public void Awake(TractorRoomComponent self) { 
        ReferenceCollector rc = self.GetParent<UI>().GameObject.GetComponent<ReferenceCollector>();
        GameObject quitButton = rc.Get<GameObject>("QuitButton");   // 退出: 退出房间,不玩了
        GameObject readyButton = rc.Get<GameObject>("ReadyButton"); // 准备:  准备开始玩儿
        GameObject multiplesObj = rc.Get<GameObject>("Multiples");
        multiples = multiplesObj.GetComponent<Text>();
        // 绑定事件
        quitButton.GetComponent<Button>().onClick.AddListener(() => { OnQuit(self).Coroutine(); });
        // readyButton.GetComponent<Button>().onClick.Add(OnReady);
        readyButton.GetComponent<Button>().onClick.AddListener(() => { OnReady(self).Coroutine(); });

        // 默认隐藏UI: ,隐藏倍率/准备按钮/牌桌(地主3张牌)
        multiplesObj.SetActive(false);
        readyButton.SetActive(false);
        rc.Get<GameObject>("Desk").SetActive(false);
        // 添加玩家面板
        GameObject gamersPanel = rc.Get<GameObject>("Gamers");
        // 【四个玩家】:上下左右,每边一个
        this.GamersPanel[0] = gamersPanel.Get<GameObject>("Left");
        this.GamersPanel[1] = gamersPanel.Get<GameObject>("Local");
        this.GamersPanel[2] = gamersPanel.Get<GameObject>("Right");
        // 添加本地玩家
        User localPlayer = ClientComponent.Instance.LocalPlayer;
        Gamer localGamer = GamerFactory.Create(localPlayer.UserID, false);
        AddGamer(localGamer, 1);
        this.GetParent<UI>().GetComponent<GamerComponent>().LocalGamer = localGamer;
    }
    // 添加玩家
    public void AddGamer(Gamer gamer, int index) {
        GetParent<UI>().GetComponent<GamerComponent>().Add(gamer, index);
        // 【游戏视图上】:每个玩家自己有个小画板,来显示每个玩家,比如自己出的牌,叫过反过的主,等,小UI 面板
        gamer.GetComponent<GamerUIComponent>().SetPanel(this.GamersPanel[index]); // 工厂生产 Gamer 的时候,会添加它相应的小画板
    }
    // 移除玩家
    public void RemoveGamer(long id) {
        Gamer gamer = GetParent<UI>().GetComponent<GamerComponent>().Remove(id);
        gamer.Dispose();
    }
    // 设置倍率: 重构游戏里,就是带不带漂
    public void SetMultiples(int multiples) {
        this.multiples.gameObject.SetActive(true);
        this.multiples.text = multiples.ToString();
    }
    // 重置倍率
    public void ResetMultiples() {
        this.multiples.gameObject.SetActive(false);
        this.multiples.text = "1";
    }
    // 退出房间
    private static async ETTask OnQuit(TractorRoomComponent self) {
        // 发送退出房间消息: 要去大厅
        self.ClientScene().GetComponent<SessionComponent>().Session.Send(new C2G_ReturnLobby_Ntt());
        // // 切换到大厅界面【不等结果吗?】也该是发布一个自定义的事件 TODO
        // Game.Scene.GetComponent<UIComponent>().Create(UIType.UILobby);
        // Game.Scene.GetComponent<UIComponent>().Remove(UIType.TractorRoom);
    }
    private static async ETTask OnReady(TractorRoomComponent self) { // 准备
        // 发送准备:  发送Actor_GamerReady_Ntt消息。 玩家加入匹配队列/退出匹配队列的逻辑均在服务端完成,客户端在不需要具体动作时都不会有变化。
        self.ClientScene().GetComponent<SessionComponent>().Session.Send(new Actor_GamerReady_Ntt());
    }
}
\end{minted}
\subsection{TractorInteractionComponent:}
\label{sec-3-6}
\subsection{}
\label{sec-3-7}

\section{【参考项目斗地主】里的源码设计相关分析:【Windows 下读源码+运行客户端】}
\label{sec-4}
\begin{itemize}
\item 这个参考项目里的源码要去 windows 里读,因为可以同时运行游戏,比较方便实时查找运行时 unity 里的控件,比直接读源码来得容易来得快。
\item 这个看今天晚上再晚点儿的时候,有没有时间去看。
\end{itemize}

\section{ET7 框架下【参考项目斗地主】的组件模块设计思路,与源码记录}
\label{sec-5}
\begin{itemize}
\item 自己是学过,有这方面的意识,但并不是说,自己就懂得,就知道该如何狠好地设计这些类。现在更多的是要受ET 框架,以及参考游戏手牌设计的启发,来帮助自己一再梳理思路,该如何设计它。
\item ET7 重构里,各组件都该是自己设计重构原项目的类的设计的必要起点。可以根据这些来系统设计重构。【活宝妹就是一定要嫁给亲爱的表哥!!!】
\item 【GamerComponent】玩家组件管理类,管理所有一个房间的玩家:是对一个房间里四个玩家的(及其在房间里的坐位位置)管理(分东南西北)。可以添加移除玩家。今天晚上来弄这一块儿吧。
\item 【Gamer】:每一个玩家
\item 【Card 牌】:有花色,和权重两个属性
\item 【拖拉机游戏房间】:多组件构成,里面嵌套一个互动组件
\item 【TractorInteractionComponent 互动组件】:几个按钮,抢不抢庄,叫不叫牌,反不反主,可是在原游戏设计里,全是鼠标的左键或是右键操作。
\end{itemize}
\section{源码分析与重构}
\label{sec-6}
\begin{itemize}
\item 还是需要相对事理一个源码里必要的关键类。因为变量太多,容易忘记。不知道哪个变量取什么值,是什么意思
\item 源码主要特点是:没有设计。像是没学过OOP/OOD 的小屁孩写的。既然今天下午是看这个项目的源码与设计重构,就可以用好电脑,要比这个舒服多了。【爱表哥,爱生活!!!活宝妹就是一定要嫁给亲爱的表哥!!!】没有分层,找不到Model 层,控制层在哪里?源码设计不功能模块化。。
\item 狠不想去读这个游戏原项目堆得山一样的源码,因为没有设计,读得会小蚂蚁掉进海量团团棉花,永远爬不出来。。。出去看球赛。晚上回来再弄这个。
\end{itemize}
% Emacs 29.1 (Org mode 8.2.7c)
\end{document}