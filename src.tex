% Created 2023-05-18 Thu 19:04
\documentclass[9pt, b5paper]{article}
\usepackage{xeCJK}
\usepackage[T1]{fontenc}
\usepackage{bera}
\usepackage[scaled]{beraserif}
\usepackage[scaled]{berasans}
\usepackage[scaled]{beramono}
\usepackage[cache=false]{minted}
\usepackage{xltxtra}
\usepackage{graphicx}
\usepackage{xcolor}
\usepackage{multirow}
\usepackage{multicol}
\usepackage{float}
\usepackage{textcomp}
\usepackage{algorithm}
\usepackage{algorithmic}
\usepackage{latexsym}
\usepackage{natbib}
\usepackage{geometry}
\geometry{left=1.2cm,right=1.2cm,top=1.5cm,bottom=1.2cm}
\usepackage[xetex,colorlinks=true,CJKbookmarks=true,linkcolor=blue,urlcolor=blue,menucolor=blue]{hyperref}
\newminted{common-lisp}{fontsize=\footnotesize} 
\author{deepwaterooo}
\date{\today}
\title{ET 框架拖拉机项目源码与设计重构}
\hypersetup{
  pdfkeywords={},
  pdfsubject={},
  pdfcreator={Emacs 28.2 (Org mode 8.2.7c)}}
\begin{document}

\maketitle
\tableofcontents


\section{DefinedConstant | CardCommands | CurrentState}
\label{sec-1}
\begin{minted}[fontsize=\scriptsize,linenos=false]{csharp}
// 程序常量
class DefinedConstant {
    // 时间常量
    internal const int FINISHEDONCEPAUSETIME = 1500; // 每圈暂停时间
    internal const int NORANKPAUSETIME = 5000; // 流局时间
    internal const int GET8CARDSTIME = 1000; // 摸8张底牌的时间
    internal const int SORTCARDSTIME = 1000; // 我的牌排序时间
    internal const int FINISHEDTHISTIME = 2500; // 每局暂停时间
    internal const int TIMERDIDA = 100; // 系统滴答
}
// 命令状态,指示下一步动作
enum CardCommands {
    ReadyCards, // 发牌命令
    DrawCenter8Cards, // 画8张底牌的命令
    WaitingForSending8Cards, // 等待扣底的命令
    DrawMySortedCards,// 排序我的牌的命令
    Pause,// 通用暂停命令
    WaitingShowPass, // 显示流局的命令
    WaitingShowBottom, // 翻底牌的命令
    WaitingForSend, // 等待出牌
    WaitingForMySending, // 等待我出牌的命令
    DrawOnceFinished,// 出完一圈后的命令
    DrawOnceRank,// 出完一局后的命令
    Undefined // 未定义的命令
}
// 保存当前游戏状态的对象
[Serializable]
struct CurrentState {
    internal int OurCurrentRank; // 自己当前的牌局
    internal int OurTotalRound; // 总轮数
    internal int OpposedCurrentRank; // 对方的牌局
    internal int OpposedTotalRound; // 总轮数

    // 当前的【庄家】:未定0, 自己1、对家2、西3、东4
    internal int Master;

    // 当前的【花色】:未定0、红桃1、黑桃2、方块3、梅花4、无主5
    internal int Suit;

    internal CardCommands CurrentCardCommands; // 当前命令
    internal CurrentState(int ourCurrentRank, int opposedCurrentRank, int suit, int master,int ourTotalRound,int opposedTotalRound, CardCommands currentCardCommands) { // tv ..
    }
}
\end{minted}
\section{CurrentPoker : Diamonds | 梅花 | 红桃 | 黑桃}
\label{sec-2}
\begin{itemize}
\item 这个东西重复四遍,什么意思嘛。就是因为这么初始化,没法支持用户配置【2 为常主】
\item 设置为一个广谱的 OneSuit 之类的类,可以实例成四种类型,并且根据2 是否为常主来设置大小
\begin{minted}[fontsize=\scriptsize,linenos=false]{csharp}
#region 方块
// 方块(2,3,4,5,6,7,8,9,10,J,Q,K,A)
        private int[] diamonds = { 0, 0, 0, 0, 0, 0, 0, 0, 0, 0, 0, 0, 0 };
        internal int[] Diamonds {
            get { return diamonds; }
            set { diamonds = value; }
        }
// 不带主的方块
        private int[] diamondsNoRank = { 0, 0, 0, 0, 0, 0, 0, 0, 0, 0, 0, 0, 0 };
        internal int[] DiamondsNoRank {
            get { return diamondsNoRank; }
            set { diamondsNoRank = value; }
        }
// 方块Rank数
        internal int DiamondsRankTotal = 0;
// 方块非Rank数
        internal int DiamondsNoRankTotal = 0;
// 排序的牌型
        internal int[] SortCards = new int[56];
#endregion // 方块
\end{minted}
\end{itemize}

\section{源码分析与重构}
\label{sec-3}
\begin{itemize}
\item 还是需要相对事理一个源码里必要的关键类。因为变量太多,容易忘记。不知道哪个变量取什么值,是什么意思
\item 源码主要特点是:没有设计。像是没学过OOP/OOD 的小屁孩写的。既然今天下午是看这个项目的源码与设计重构,就可以用好电脑,要比这个舒服多了。【爱表哥,爱生活!!!活宝妹就是一定要嫁给亲爱的表哥!!!】没有分层,找不到Model 层,控制层在哪里?源码设计不功能模块化。。
\end{itemize}
% Emacs 28.2 (Org mode 8.2.7c)
\end{document}